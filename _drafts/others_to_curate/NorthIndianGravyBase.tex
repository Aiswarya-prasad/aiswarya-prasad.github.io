This is the first step that results in a gravy that can then be adapted into many other recipies with or without mior modifications. For example, cooked moong dal can be directly mixed with this base for a quick dal dish to go with rice or chapati.

\large{\textbf{Ingredients}}
% \begin{multicols}{2}
  \large{\textbf{Common}}
  \begin{itemize}
    \item Oil \quad\quad\quad\quad \textit{X}
    \item Onions \quad\quad\quad\quad \textit{X (Chopped)}
    \item Tomato (crushed and canned or fresh and chopped ) \quad\quad\quad\quad \textit{X}
  \end{itemize}
  % \columnbreak
  \large{\textbf{Indian/special}}
  \begin{itemize}
    \item Garam masala \quad\quad\quad\quad \textit{X tablespoons}
    \item Red chilli powder \quad\quad\quad\quad \textit{X tablespoons}
    \item Turmeric powder \quad\quad\quad\quad \textit{X tablespoons}
    \item Jeera (Cumin seeds) \quad\quad\quad\quad \textit{X tablespoons}
  \end{itemize}
% \end{multicols}

\large{\textbf{Recipe}}

\begin{enumerate}
  \item Heat up a few teaspoons for oil and add a teaspoon of jeera to it.
  \item Optionally, depending on the flavour profile you would like, you could add other whole spices such as dalchini (cinnamon), cardamom, star anise, and cloves.
\end{enumerate}

\large{\textbf{Suggested pairing}}
