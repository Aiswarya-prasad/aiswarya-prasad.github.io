
\large{\textbf{Ingredients}}
% \begin{multicols}{2}
  \large{\textbf{Common}}
  \begin{itemize}
    \item Brinjal \quad\quad\quad\quad \textit{X (Chopped)}
    \item Onions \quad\quad\quad\quad \textit{X (Chopped)}
  \end{itemize}
  % \columnbreak
  \large{\textbf{Indian/special}}
  \begin{itemize}
    \item Gingely oil \quad\quad\quad\quad \textit{X tablespoons}
    \item Urad dal \quad\quad\quad\quad \textit{X grams}
    \item Curry leaves \quad\quad\quad\quad \textit{X leaves}
    \item Black mustard seeds \quad\quad\quad\quad \textit{X teaspoons}
  \end{itemize}
% \end{multicols}

\large{\textbf{Recipe}}

\begin{enumerate}
  \item Heat up a few tablespoons of oil (use gingely oil for a nice flavour - not other kinds of sesame oil - otherwise use a neutral oi)
  \item In the oil, fry up a teaspoon of black mustard seeds and 2 - 3 teaspoon of urad dal
  \item Wash and add 4 - 5 curry leaves roughly torn apart
  \item Fry until the urad dal changes color slightly but do not let it get too dark
  \item 1 large sized or 2 small onion(s) into small cubes and add to the oil
  \item Add 1 teaspoon of sambar powder, salt and 1 teaspoon of coriander powder
  \item Chop the brinjal into large cubes or if they the are the small round variety, either cut them into quaters or split them into quaters and fill them with the fried onions
  \item Add a splash of water and cook covered until the brinjals turn soft
  \item Remove the cover and continue cooking until they get dry and roasted
\end{enumerate}

\large{\textbf{Suggested pairing}}

Serve as a side with kootu, sambar and rice or with curd rice.
